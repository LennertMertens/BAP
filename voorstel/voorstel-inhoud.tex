%---------- Inleiding ---------------------------------------------------------

\section{Introductie} % The \section*{} command stops section numbering
\label{sec:introductie}
Serverless infrastructuur is alomtegenwoordig en verschillende bedrijven beginnen met het ontdekken van deze technologie. De technologie, die deel uitmaakt van Cloud Computing, nl. FaaS (Function as a Service)  biedt nieuwe mogelijkheden voor ontwikkelaars. Bedrijven zijn reeds bezig met de adoptie van deze technologie en grote spelers zoals Amazon, Microsoft en Google proberen daarop in te spelen door serverless diensten aan te bieden. De bedrijven die op dit moment de diensten aanbieden, zijn reeds welbekende Cloud providers daarnaast zien we ook een opmars in de open-source wereld waarin verschillende ontwikkelaars bezig zijn met het ontwikkelen van open-source serverless frameworks.


De serverless benadering (FaaS) is nog volop in ontwikkeling en nog heel recent, de meeste relevante bronnen omtrent serverless zijn niet ouder dan twee jaar wat er dus op wijst dat dit in de toekomst nog aan grote populariteit kan winnen. Om een beter begrip van serverless te geven samen met de mogelijkheden dat het met zich meebrengt is het zinvol hierover een onderzoek te voeren. 


Dit onderwerp werd in samenwerking met het stagebedrijf besproken en uit verdere interesse in deze nieuwe technologie wordt hierop verder gewerkt. De huidige serverless providers werken nu allemaal in een Cloud gebaseerde omgeving, het lijkt interessant om te onderzoeken of hier alternatieven voor zijn. Sommige bedrijven kiezen er vandaag bewust nog voor om niets van resources te migreren naar de Cloud om allerhande redenen, doch willen zij de voordelen van FaaS ook kunnen toepassen in ontwikkeling. De mogelijkheid om op locatie of in een private cloud een FaaS infrastructuur op te zetten is een interessant onderzoek voor zowel het bedrijf als de student. Met een Proof-of-Concept wordt dit soort infrastructuur opgezet om een hands-on demo te kunnen voorzien dat meer inzicht geeft in wat er al dan niet mogelijk is.


De doelstelling is zoals eerder vermeld een volledig onderzoek naar de mogelijkheden van serverless infrastructuren op locatie of in een private cloud. Het onderzoek bestaat uit een een theoretische benadering die gestaafd wordt door een proof-of-concept waarin onderzochte tools en frameworks in gedemonstreerd worden.


\textbf{Onderzoeksvraag: 
    Wat omvat serverless infrastructuren en welke mogelijke open-source projecten bestaan als alternatief voor de huidige serverless infrastructuren die Cloud providers aanbieden?}
\begin{itemize}
  \item Wat is een serverless infrastructuur?
  \item Wat zijn de voor- en nadelen van een serverless infrastructuur?
  \item Wat is het verschil met traditionele infrastructuur?
  \item Welke technologieën maken serverless infrastructuur mogelijk?
  \item Welke open-source alternatieven bestaan er?
  \item Wat is het verschil tussen traditionele en next-gen applicaties?
  \item Welke voordelen bieden next-gen applicaties?
  \item Rol van containers in het verhaal?
  \item Rol van microservices?
  \item Waarom kan de huidige infrastructuur hier al dan niet voor instaan of waarom zouden we beter migreren naar een serverless oplossing?
  \item Waarom moeten IT-bedrijven naar serverless architecturen of next-gen applicaties evolueren?
\end{itemize}

%---------- Stand van zaken ---------------------------------------------------

\section{State-of-the-art}
\label{sec:state-of-the-art}

Serverless is een technologie die in opmars is en verschillende IT-bedrijven springen nu volledig op dit concept. Alvorens in detail op dit onderwerp in te gaan is het belangrijk om een algemeen beeld te hebben hierover. Deze sectie handelt over wat serverless en FaaS inhoudt, wat de voordelen/nadelen zijn, welke bedrijven reeds bezig zijn met deze concepten en welke benaderingen er in ontwikkeling zijn.

\subsection{Wat is Serverless?}
Serverless is een computing technologie dat volledige controle over infrastructuur geeft aan de cloud provider. De provider staat in voor het volledige management van containers waarop ontwikkelaars functies uit kunnen voeren. Door dit te doen zorgt deze architectuur ervoor dat het niet meer nodig is om systemen voortdurend te laten draaien en enkel te laten werken als er event-driven taken moeten worden uitgevoerd. Deze architectuur laat het toe op een eenvoudige manier een schaalbare applicatie uit te rollen. Serverless stelt ontwikkelaars in staat om code te schrijven, op te laden en uit te voeren zonder enige zorg over de onderliggende infrastructuur. Deze technologie kent talloze mogelijkheden die daarenboven ontwikkeling van applicaties tot tweemaal sneller kan maken. \autocite{Stigler2017} Een andere benaming die vaak terugkomt in serverless computing is FaaS oftewel \textit{Function as a Service} dit wordt door \textcite{VanEyck2018} omschreven als een vorm van serverless computing waar de cloud provider instaat voor het management van de resources, de levenscyclus en de event-driven uitvoering van functies die door de gebruiker voorzien werden.

\subsection{Voordelen}
Werken met een serverless omgeving zoals AWS Lambda kan volgens \textcite{Perez2018} de kost van resources voor 70\% verminderen naargelang het aantal gebruikers stijgt met garantie op even goede performance. Serverless biedt de mogelijkheid om snel uit te breiden wanneer dit nodig is zonder dat hier performantie van afhankelijk is. Grote bedrijven zoals Netflix en LinkedIn werken tegenwoordig met microservices in een serverless omgeving, dit verhoogt de flexibiliteit en reduceert de kost aanzienlijk. Daarnaast sluit deze manier van werken aan bij de agile benadering gehanteerd bij softwareontwikkeling. \autocite{Villamizar2017} Een van de grote voordelen wanneer klanten opteren voor serverless computing is dat ze enkel hoeven te betalen voor de code die effectief uitgevoerd wordt. Wanneer er meer code moet uitgevoerd worden dan laat de provider het toe om op te schalen zonder dat de ontwikkelaar op voorhand rekening moet houden hoeveel geheugen en cpu-tijd nodig zal zijn. De ontwikkelaars kunnen op deze manier focussen op code en niet zozeer op de infrastructuur waarop de code draait, deze verantwoordelijkheid verschuift volledig naar de provider. \autocite{Savage2018}

\subsection{Benaderingen}
Bij het raadplegen van bronnen wordt er vaak over grote cloud providers gesproken zoals AWS, Google Cloud en Microsoft Azure, al deze bedrijven voorzien serverless infrastructuren in de publieke cloud. Een infrastructuur die onderhouden wordt door één van voorgaande kan dus nergens anders uitgerold worden buiten in de publieke cloud. Bedrijven die gebruik willen maken van FaaS infrastructuren op locatie of in de private cloud kunnen gebruik maken van open-source alternatieven. Mogelijke alternatieven voor een serverless infrastructuur zijn OpenFaaS \autocite{Ellis2017} en het Serverless Framework \autocite{Serverless2018}. Naar deze mogelijke alternatieven worden

% Voor literatuurverwijzingen zijn er twee belangrijke commando's:
% \autocite{KEY} => (Auteur, jaartal) Gebruik dit als de naam van de auteur
%   geen onderdeel is van de zin.
% \textcite{KEY} => Auteur (jaartal)  Gebruik dit als de auteursnaam wel een
%   functie heeft in de zin (bv. ``Uit onderzoek door Doll & Hill (1954) bleek
%   ...'')


%---------- Methodologie ------------------------------------------------------
\section{Methodologie}
\label{sec:methodologie}

Allereerst wordt er aan deskresearch gedaan om de nodige informatie omtrent het onderwerp te verzamelen, deze bronnen worden verwerkt in een literatuurstudie die als basis voor dit onderzoek fungeert. Aan de hand van de verzamelde bronnen worden er antwoorden geformuleerd op de theoretische deelonderzoeksvragen. Wanneer er voldoende bronnen zijn verzameld wordt er een laboratoriumonderzoek gevoerd waarin een proof-of-concept wordt uitgewerkt. In deze proof-of-concept wordt er een full stack uitgewerkt met tools zoals: Docker, eventueel Docker Swarm of Kubernetes, eventuele automatiseringstools en een open-source framework voor het opzetten van een FaaS omgeving. Op de opgezette omgeving wordt ook een applicatie uitgerold.

%---------- Verwachte resultaten ----------------------------------------------
\section{Verwachte resultaten}
\label{sec:verwachte_resultaten}

Er wordt verwacht dat deze paper een significante bijdrage kan leveren aan bedrijven die geïnteresseerd zijn in het opzetten van een serverless omgeving. De paper stelt begunstigden in staat een algemeen beeld te krijgen van wat serverless is, wat er bestaat en met welke alternatieven. Er wordt ook gedemonstreerd hoe er zelf een demo-omgeving kan worden opgezet. Op basis van dit document moeten bedrijven een beslissing kunnen nemen of het voor hen zinvol is te evolueren naar serverless infrastructuren.


%---------- Verwachte conclusies ----------------------------------------------
\section{Verwachte conclusies}
\label{sec:verwachte_conclusies}

Er wordt verwacht dat er verschillende open-source alternatieven zijn voor serverless infrastructuren. De demo-opstelling maakt het mogelijk om verschillende benaderingen te vergelijken en geïnteresseerde bedrijven bij te staan bij de keuze om eventueel te evolueren naar serverless infrastructuren.
