%%=============================================================================
%% Samenvatting
%%=============================================================================

% TODO: De "abstract" of samenvatting is een kernachtige (~ 1 blz. voor een
% thesis) synthese van het document.
%
% Deze aspecten moeten zeker aan bod komen:
% - Context: waarom is dit werk belangrijk?
% - Nood: waarom moest dit onderzocht worden?
% - Taak: wat heb je precies gedaan?
% - Object: wat staat in dit document geschreven?
% - Resultaat: wat was het resultaat?
% - Conclusie: wat is/zijn de belangrijkste conclusie(s)?
% - Perspectief: blijven er nog vragen open die in de toekomst nog kunnen
%    onderzocht worden? Wat is een mogelijk vervolg voor jouw onderzoek?
%
% LET OP! Een samenvatting is GEEN voorwoord!

%%---------- Nederlandse samenvatting -----------------------------------------
%
% TODO: Als je je bachelorproef in het Engels schrijft, moet je eerst een
% Nederlandse samenvatting invoegen. Haal daarvoor onderstaande code uit
% commentaar.
% Wie zijn bachelorproef in het Nederlands schrijft, kan dit negeren, de inhoud
% wordt niet in het document ingevoegd.
\chapter*{Samenvatting}
Serverless computing is alomtegenwoordig en kent een grote opmars binnen de IT wereld. Verschillende bedrijven hebben deze technologie reeds geadopteerd en in de toekomst zullen er nog heel veel volgen. Omdat de strijd binnen de serverless wereld nog niet gestreden is en er nog geen standaard is gezet, is een onderzoek naar veelbelovende open source serverless oplossingen zinvol. Nubera is op zoek naar een open source serverless framework, dat aanleunt bij hun requirements, om aan te bieden bij klanten die serverless willen gaan werken. Nubera gaf aan interesse te hebben in Fission en zocht hiervoor een veelbelovend alternatief. 
\\\\
Vertrekkend vanuit een gedetailleerde literatuurstudie die de theorie omtrent serverless en alle bijkomstigheden behandelt werden in samenspraak met Nubera requirements gecapteerd. Op basis van de requirements wordt in de uitwerking, aan de hand van een oplijsting, een interessant open source framework gekozen en vergeleken met Fission. OpenFaaS blijkt het beste alternatief te zijn voor Fission conform aan de opgestelde requirements. Beide frameworks worden aan de hand van een Proof of Concept met elkaar vergeleken om hieruit een aanbeveling voor Nubera te kunnen vormen. 
\\\\
Vanuit de Proof of Concept en vergelijkende studie komt OpenFaaS uit als het meest veelbelovend open source serverless framework dat aanleunt bij de requirements van Nubera.
\\\\
Omdat serverless computing nog in de kinderschoenen staat en er heel wat open source frameworks in ontwikkeling zijn is het zinvol onderzoek te doen naar anderen en te vergelijken met de frameworks die in dit onderzoek werden behandeld. Deze bachelorproef vormt een aanzet zijn voor een vergelijkende studie tussen meer open source serverless frameworks.
