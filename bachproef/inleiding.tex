%%=============================================================================
%% Inleiding
%%=============================================================================

\chapter{Inleiding}
\label{ch:inleiding}

%%De inleiding moet de lezer net genoeg informatie verschaffen om het onderwerp te begrijpen en in te zien waarom de onderzoeksvraag de moeite waard is om te onderzoeken. In de inleiding ga je literatuurverwijzingen beperken, zodat de tekst vlot leesbaar blijft. Je kan de inleiding verder onderverdelen in secties als dit de tekst verduidelijkt. Zaken die aan bod kunnen komen in de inleiding~\autocite{Pollefliet2011}:%%
Cloud Computing technologie is alomtegenwoordig, er bestaan verschillende benaderingen met elk hun specifieke eigenschappen, voor-en nadelen. Meer en meer bedrijven zijn reeds begonnen met migratie van resources naar de Cloud en dit biedt een variëteit aan mogelijkheden. Grote Cloud providers zoals Google Cloud Platform, Amazon Web Services (AWS), Microsoft Azure en IBM bieden hun klanten al heel wat mogelijkheden aan om gebruik te maken van Cloud diensten. Vele bedrijven verkiezen toch nog steeds een eigen infrastructuur boven het toevertrouwen van resources aan grote Cloud providers zoals hierboven genoemd. Klassieke bedrijven waar informatie confidentieel is hebben toch enige argwaan om zomaar alles in de publieke Cloud te draaien. Bedrijven die enige afstand willen nemen van de publieke Cloud kunnen toch meegenieten van de veelzijdige mogelijkheden dat Cloud computing met zich meebrengt. Het is mogelijk voor bedrijven om gebruik te maken van software zoals VMWare, Microsoft Cloud, IBM Bluemix Cloud om zo een eigen Cloud infrastructuur op te zetten op locatie of in een private Cloud. Deze benadering biedt bedrijven, die mee willen groeien in het hele Cloud gebeuren maar niet willen migreren naar de publieke Cloud, toch de mogelijkheid mee te evolueren.
\newline
\newline
De jongste jaren is er een nieuwe trend in Cloud computing geëvolueerd, namelijk Function as a Service (FaaS), beter gekend als serverless. Deze technologie heeft heel wat voordelen bij voortdurende deployment van next-generation applicaties. Serverless neemt overhead weg bij softwareontwikkelaars en stelt hun in staat enkel te focussen op wat er voor hen het meest toe doet, namelijk de software zelf.
\newline
\newline
Deze studie behandelt een uiteenzetting van enkele basis Cloud computing principes, een onderzoek naar wat serverless precies inhoudt, wat de voor- en nadelen zijn en dit alles wordt gestaafd met een proof-of-concept.

%%\begin{itemize}
%%  \item context, achtergrond
%%  \item afbakenen van het onderwerp
%%  \item verantwoording van het onderwerp, methodologie
%%  \item probleemstelling
%%  \item onderzoeksdoelstelling
%%  \item onderzoeksvraag
%%  \item \ldots
%%\end{itemize}

\section{Probleemstelling}
\label{sec:probleemstelling}

%%Uit je probleemstelling moet duidelijk zijn dat je onderzoek een meerwaarde heeft voor een concrete doelgroep. De doelgroep moet goed gedefinieerd en afgelijnd zijn. Doelgroepen als ``bedrijven,'' ``KMO's,'' systeembeheerders, enz.~zijn nog te vaag. Als je een lijstje kan maken van de personen/organisaties die een meerwaarde zullen vinden in deze bachelorproef (dit is eigenlijk je steekproefkader), dan is dat een indicatie dat de doelgroep goed gedefinieerd is. Dit kan een enkel bedrijf zijn of zelfs één persoon (je co-promotor/opdrachtgever).%%

Er zijn bedrijven die interesse hebben in het gebruik van serverless omgevingen maar opteren om deze niet te draaien bij grote Cloud providers zoals Google, Amazon of Microsoft Azure. Bedrijven die serverless willen werken in hun eigen datacenter, de private cloud of op locatie hebben nood aan een framework dat hen in staat stelt dit te doen. Nubera heeft klanten die argwanend zijn wanneer ze voorstellen een Cloud omgeving op te zetten bij een van de grote spelers. Omdat nubera momenteel nog geen oplossing gebruikt om serverless te implementeren bij klanten op locatie werd mij gevraagd hiernaar onderzoek te doen en een reproduceerbare proof-of-concept uit te werken. Deze bachelorproef dient als gids om serverless te werken.

\section{Onderzoeksvraag}
\label{sec:onderzoeksvraag}

%%Wees zo concreet mogelijk bij het formuleren van je onderzoeksvraag. Een onderzoeksvraag is trouwens iets waar nog niemand op dit moment een antwoord heeft (voor zover je kan nagaan). Het opzoeken van bestaande informatie (bv. ``welke tools bestaan er voor deze toepassing?'') is dus geen onderzoeksvraag. Je kan de onderzoeksvraag verder specifiëren in deelvragen. Bv.~als je onderzoek gaat over performantiemetingen, dan%%

Onderzoeksvraag: 
\begin{itemize}
    \item Wat omvat serverless infrastructuren en welke mogelijke open-source projecten bestaan er als alternatief voor de huidige serverless infrastructuren die Cloud providers aanbieden?
\end{itemize}

Deelonderzoeksvragen: 
\begin{itemize}
    \item Wat is Cloud computing?
    \item Wat is serverless?
    \item Wat is de rol van onderliggende componenten (containers, microservices)?
    \item Wat zijn de voor-en nadelen van serverless infrastructuur?
    \item Wat is het verschil met traditionele infrastructuren?
    \item Welke technologieën maken serverless infrastructuren mogelijk?
\end{itemize}

Voorwaarden: 
\begin{itemize}
    \item Het onderdeel stand van zaken vormt een begrijpbare uiteenzetting van enkele basisconcepten en geeft een breed beeld over serverless en alle bijkomstigheden
    \item Onderzoek wordt gestaafd door een proof-of-concept met een interessant open-source framework dat voor Nubera een meerwaarde kan bieden
    \item Het opzetten van de omgeving is reproduceerbaar en is duidelijk gedocumenteerd
\end{itemize}



\section{Onderzoeksdoelstelling}
\label{sec:onderzoeksdoelstelling}

%%Wat is het beoogde resultaat van je bachelorproef? Wat zijn de criteria voor succes? Beschrijf die zo concreet mogelijk.%%
Deze bachlorproef moet kunnen fungeren als een rode draad voor het implementeren van een serverless open-source oplossing voor bedrijven die dit willen opzetten op locatie of in de private Cloud. Deze gids heeft een meerwaarde voor Nubera omdat ze dit als handleiding kunnen gebruiken voor implementatie bij klanten. Het onderzoek moet voldoende inzichten geven zodat de volledige bachelorproef voor iedereen begrijpbaar is, ook voor mensen met weinig Cloud computing ervaring.


\section{Opzet van deze bachelorproef}
\label{sec:opzet-bachelorproef}

% Het is gebruikelijk aan het einde van de inleiding een overzicht te
% geven van de opbouw van de rest van de tekst. Deze sectie bevat al een aanzet
% die je kan aanvullen/aanpassen in functie van je eigen tekst.

De rest van deze bachelorproef is als volgt opgebouwd:

In Hoofdstuk~\ref{ch:stand-van-zaken} wordt een overzicht gegeven van de stand van zaken binnen het onderzoeksdomein, op basis van een literatuurstudie.

In Hoofdstuk~\ref{ch:methodologie} wordt de methodologie toegelicht en worden de gebruikte onderzoekstechnieken besproken om een antwoord te kunnen formuleren op de onderzoeksvragen.

% TODO: Vul hier aan voor je eigen hoofstukken, één of twee zinnen per hoofdstuk

In Hoofdstuk~\ref{ch:conclusie}, tenslotte, wordt de conclusie gegeven en een antwoord geformuleerd op de onderzoeksvragen. Daarbij wordt ook een aanzet gegeven voor toekomstig onderzoek binnen dit domein.

