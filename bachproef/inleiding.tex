%%=============================================================================
%% Inleiding
%%=============================================================================

\chapter{Inleiding}
\label{ch:inleiding}

%%De inleiding moet de lezer net genoeg informatie verschaffen om het onderwerp te begrijpen en in te zien waarom de onderzoeksvraag de moeite waard is om te onderzoeken. In de inleiding ga je literatuurverwijzingen beperken, zodat de tekst vlot leesbaar blijft. Je kan de inleiding verder onderverdelen in secties als dit de tekst verduidelijkt. Zaken die aan bod kunnen komen in de inleiding~\autocite{Pollefliet2011}:%%
Cloud computing technologie is alomtegenwoordig, er bestaan verschillende benaderingen met elk hun specifieke eigenschappen, voor- en nadelen. Meer en meer bedrijven zijn reeds begonnen met migratie van resources naar de cloud en dit biedt een variëteit aan mogelijkheden. Grote cloud providers zoals Google Cloud Platform, Amazon Web Services (AWS), Microsoft Azure en IBM bieden hun klanten al heel wat mogelijkheden aan om gebruik te maken van clouddiensten. Vele bedrijven verkiezen toch nog steeds een eigen infrastructuur boven het toevertrouwen van resources aan grote cloud providers zoals hierboven genoemd. Klassieke bedrijven waar informatie confidentieel is, hebben toch enige argwaan om zomaar alles in de publieke cloud te draaien. De angst die bestaat in migratie naar de cloud is te wijten aan verschillende feiten die al aan het ligt zijn gekomen waarbij bijvoorbeeld een Google inzage heeft in delicate gegevens van zijn klanten. Bedrijven die enige afstand willen nemen van de publieke cloud kunnen toch meegenieten van de veelzijdigheid dat cloud computing met zich meebrengt. Het is mogelijk voor bedrijven om gebruik te maken van software zoals VMWare, Microsoft Cloud, IBM Bluemix Cloud om zo een eigen cloud infrastructuur in house op te zetten, dit soort omgeving draait in tegenstelling tot grote providers dus niet in publieke datacentersmaar wel in private omgevingen die meestal beheerd worden door het bedrijf zelf. Deze benadering biedt bedrijven, die mee willen groeien in het hele cloud gebeuren maar niet willen migreren naar de publieke cloud, toch de mogelijkheid mee te evolueren.
\\\\

De jongste jaren is er een nieuwe trend in Cloud computing geëvolueerd, namelijk serverless of vaak gerefereerd als Function-as-a-Service. Serverless technologie biedt heel wat voordelen op basis van verschillende aspecten. Het voornaamste voordeel van serverless computig is dat het overhead wegneemt bij softwareontwikkelaars en hun in staat stelt enkel te focussen op wat er voor hen het meest toe doet, namelijk de software zelf.
\\\\

Deze studie behandelt enerzijds een uiteenzetting van de basis cloud computing principes. Er wordt ook onderzoek gedaan naar wat serverless precies inhoudt, wat de voor- en nadelen zijn en welke componenten serverless mogelijk maken. De theoretische omschrijving van dit concept wordt geïllustreerd aan de hand van voorbeelden om zo de architectuur op een eenvoudige manier over te brengen. Daarnaast wordt op basis van verschillende requirements een overweging gemaakt voor het kiezen van twee open-source frameworks die de mogelijkheden aanbieden serverless te draaien in house. Het onderzoek wordt gestaafd met een proof-of-concept die twee frameworks vergelijkt.

%%\begin{itemize}
%%  \item context, achtergrond
%%  \item afbakenen van het onderwerp
%%  \item verantwoording van het onderwerp, methodologie
%%  \item probleemstelling
%%  \item onderzoeksdoelstelling
%%  \item onderzoeksvraag
%%  \item \ldots
%%\end{itemize}

\section{Probleemstelling}
\label{sec:probleemstelling}

%%Uit je probleemstelling moet duidelijk zijn dat je onderzoek een meerwaarde heeft voor een concrete doelgroep. De doelgroep moet goed gedefinieerd en afgelijnd zijn. Doelgroepen als ``bedrijven,'' ``KMO's,'' systeembeheerders, enz.~zijn nog te vaag. Als je een lijstje kan maken van de personen/organisaties die een meerwaarde zullen vinden in deze bachelorproef (dit is eigenlijk je steekproefkader), dan is dat een indicatie dat de doelgroep goed gedefinieerd is. Dit kan een enkel bedrijf zijn of zelfs één persoon (je co-promotor/opdrachtgever).%%

Er zijn bedrijven die interesse hebben in het gebruik van serverless omgevingen, maar opteren om deze niet te draaien bij grote cloud providers zoals Google, Amazon of Microsoft Azure. Bedrijven die serverless willen werken in hun eigen datacenter, de private cloud of op locatie hebben nood aan een framework dat hen in staat stelt dit te verwezenlijken. Nubera heeft klanten die argwanend zijn wanneer ze voorstellen een cloud omgeving op te zetten bij een van de grote spelers. Omdat Nubera momenteel nog geen oplossing gebruikt om serverless te implementeren bij klanten in house, werd mij gevraagd hiernaar onderzoek te doen en twee mogelijke frameworks te behandelen die voldoen aan vooropgestelde requirements. Er wordt een reproduceerbare proof-of-concept uitgewerkt die inzicht geeft in de behandelde frameworks.

\section{Onderzoeksvraag}
\label{sec:onderzoeksvraag}

%%Wees zo concreet mogelijk bij het formuleren van je onderzoeksvraag. Een onderzoeksvraag is trouwens iets waar nog niemand op dit moment een antwoord heeft (voor zover je kan nagaan). Het opzoeken van bestaande informatie (bv. ``welke tools bestaan er voor deze toepassing?'') is dus geen onderzoeksvraag. Je kan de onderzoeksvraag verder specifiëren in deelvragen. Bv.~als je onderzoek gaat over performantiemetingen, dan%%

Onderzoeksvraag: 
\begin{itemize}
    \item Hoe werkt een serverless omgeving in house: vergelijking tussen twee serverless frameworks  voor serverless in house infrastructuur, welk framework neemt de voorkeur?
\end{itemize}

Deelonderzoeksvragen: 
\begin{itemize}
    \item Welke voordelen biedt een open-source in house serverless oplossing?
    \item Waarom zou een bedrijf kiezen voor een in house serverless infrastructuur?
    \item Waarin verschillen de open-source frameworks van elkaar?
    \item Op basis van wat kunnen de frameworks worden onderscheiden? 
\end{itemize}

Voorwaarden: 
\begin{itemize}
    \item Het onderdeel stand van zaken vormt een begrijpbare uiteenzetting van enkele basisconcepten en geeft een breed beeld over serverless en alle bijkomstigheden.
    \item Onderzoek wordt gestaafd door een proof-of-concept van  twee interessante open-source frameworks die  Nubera zou kunnen gebruiken in serverless oplossingen in house.
    \item Het opzetten van de omgeving is reproduceerbaar en is duidelijk gedocumenteerd.
\end{itemize}



\section{Onderzoeksdoelstelling}
\label{sec:onderzoeksdoelstelling}

%%Wat is het beoogde resultaat van je bachelorproef? Wat zijn de criteria voor succes? Beschrijf die zo concreet mogelijk.%%
Deze bachelorproef moet kunnen fungeren als een rode draad voor het evolueren naar serverless infrastructuur. Het onderzoek biedt inzicht in open-source serverless oplossingen voor bedrijven die opteren om gebruik te maken van eigen infrastructuur of de private cloud. Deze gids heeft een meerwaarde voor Nubera omdat het op basis van verschillende requirements een overweging kunnen maken welk framework het interessantst is. Het onderzoek moet voldoende inzichten geven zodat de volledige bachelorproef voor iedereen begrijpbaar is, ook voor mensen met weinig cloud computing ervaring.


\section{Opzet van deze bachelorproef}
\label{sec:opzet-bachelorproef}

% Het is gebruikelijk aan het einde van de inleiding een overzicht te
% geven van de opbouw van de rest van de tekst. Deze sectie bevat al een aanzet
% die je kan aanvullen/aanpassen in functie van je eigen tekst.

Het vervolg van deze bachelorproef is als volgt opgebouwd:

In Hoofdstuk~\ref{ch:stand-van-zaken} wordt een overzicht gegeven van de stand van zaken binnen het onderzoeksdomein, op basis van een literatuurstudie.

In Hoofdstuk~\ref{ch:methodologie} wordt de methodologie toegelicht en worden de gebruikte onderzoekstechnieken besproken om een antwoord te kunnen formuleren op de onderzoeksvragen.

% TODO: Vul hier aan voor je eigen hoofstukken, één of twee zinnen per hoofdstuk
In Hoofdstuk~\ref{ch:uitwerking} worden op basis van enkele requirements, opgelegd door Nubera, twee interessante open-source frameworks gekozen voor het draaien van een serverless infrastructuur on-premises of in de private cloud.


In Hoofdstuk~\ref{ch:conclusie}, tenslotte, wordt de conclusie gegeven en een antwoord geformuleerd op de onderzoeksvragen. Daarbij wordt ook een aanzet gegeven voor toekomstig onderzoek binnen dit domein.

