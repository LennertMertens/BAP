%%=============================================================================
%% Conclusie
%%=============================================================================

\chapter{Conclusie}
\label{ch:conclusie}

%% TODO: Trek een duidelijke conclusie, in de vorm van een antwoord op de
%% onderzoeksvra(a)g(en). Wat was jouw bijdrage aan het onderzoeksdomein en
%% hoe biedt dit meerwaarde aan het vakgebied/doelgroep? Reflecteer kritisch
%% over het resultaat. Had je deze uitkomst verwacht? Zijn er zaken die nog
%% niet duidelijk zijn? Heeft het onderzoek geleid tot nieuwe vragen die
%% uitnodigen tot verder onderzoek?

Het gevoerde onderzoek biedt inzicht in alle aspecten omtrent serverless computing waarin Nubera geïnteresseerd is. Alle requirements die werden opgesteld kwamen aan bod en werden vergeleken tussen twee open source frameworks. Daarnaast biedt hoofdstuk~\ref{ch:stand-van-zaken} een theoretische uiteenzetting van het serverless plaatje dat medewerkers en klanten van Nubera warm maakt voor het evolueren naar serverless applicaties.
\\\\
Uit de experimenten blijkt het ene framework beter te scoren dan het andere op basis van enkele vergelijkingen die gevoerd werden in hoofdstuk~\ref{ch:vergelijking-frameworks}. Uit de experimenten blijkt dat Fission beter presteert op basis van executietijd, de demofuncties die op het Fission framework werden gedeployed en aangeroepen hebben een kortere uitvoeringstijd dan de functies op OpenFaaS.
\\\\
OpenFaaS biedt de aangenaamste gebruikerservaring voor gebruikers die nog niet vertrouwd zijn met serverless frameworks. De UI die OpenFaaS standaard meelevert in de installatie is eveneens gebruiksvriendelijk en eenvoudig in gebruik. De interface is intuïtief en vereist eigenlijk geen ervaring met OpenFaaS om hiermee aan de slag te kunnen. Via de UI kunnen functies worden beheerd, gecreëerd en aangeroepen. Standaard zijn er via de interface een aantal community functies beschikbaar die rechtstreeks kunnen worden gedeployed. OpenFaaS voorziet YAML templates voor het definiëren van functies, dit is een overzichtelijke manier om functies te deployen. In tegenstelling tot OpenFaaS voorziet Fission op dit moment nog geen gebruiksvriendelijke UI of YAML templates voor functies. Het vertrouwd raken met OpenFaaS verliep vlotter dan bij Fission dankzij de betere documentatie en de waaier aan extra blogs en video's die hierover te vinden zijn. OpenFaaS is volgens de gevoerde experimenten een duidelijke winnaar op vlak van gebruiksvriendelijkheid. 
\\\\
Beide frameworks zijn vrij makkelijk te installeren maar OpenFaaS voorziet standaard meer componenten. Bij OpenFaaS werkt autoscaling zonder dat er nog extra zaken zoals metrics-server moeten worden geïnstalleerd in tegenstelling tot Fission. Als beginner is het veel makkelijker van start te gaan met OpenFaaS omdat dit alle componenten bevat die nodig zijn voor een production ready serverless omgeving. Het OpenFaaS framework werkt ''out-of-the-box'' zonder dat er diepgaande kennis van onderliggende infrastructuur nodig is. Fission verwacht wel enig inzicht in de onderliggende infrastructuur en dit schrikt gebruikers die nieuw zijn met het onderwerp mogelijks af.
\\\\
Het gebruik van open source frameworks kent enkele grote voordelen. Gebruikers zijn niet afhankelijk van een cloud provider die de serverless diensten aanbiedt zoals Amazon, Google of Microsoft. Vendor lock-in wordt dus uitgeschakeld dus gebruikers zijn niet meer gebonden aan een derde partij. Open source frameworks kunnen gedeployed worden op een eigen infrastructuur in een privaat datacenter on-premises of in een private cloudomgeving. Het gebruik van FaaS oplossingen beperkt het resource verbruik en voorziet autoscaling van functies zonder dat er rekening moet gehouden worden met onderliggende infrastructuur. De frameworks die werden behandeld kunnen ook worden gedeployed op een bestaande Kubernetescluster waardoor er geen nood is aan extra infrastructuur of virtuele servers.
\\\\
In het onderzoek werd getracht het meest interessante open source serverless framework conform aan de vereisten van Nubera te vinden. Op basis van literair onderzoek, uitgevoerde experimenten en resultatenverwerking werd er tot een besluit gekomen. Het meest veelbelovende framework voor Nubera is OpenFaaS. Zoals reeds werd aangehaald, ligt de uitvoeringstijd van functies lager bij OpenFaaS dan bij Fission maar het verschil legt geen belemmeringen op. Nubera is op zoek naar een serverless framework dat enige maturiteit heeft en goed onderhouden wordt. Daarnaast voldoet OpenFaaS aan alle opgelegde requirements en konden alle functionaliteiten eveneens worden gedemonstreerd. Het onderzoek naar OpenFaaS verliep vlot gezien de duidelijke documentatie en de blogposts die online terug te vinden zijn rond het framework. Aan de ''Nice-to-have'' requirements werd ook voldaan. Klanten van Nubera die interesse hebben in het framework hebben bij OpenFaaS de mogelijkheid om gebruik te maken van de UI, dit is tot op de dag van vandaag iets waar heel veel bedrijven belang aan hechten. De UI zal klassieke klanten kunnen overtuigen om ook aan de slag te gaan met dit serverless framework. Het Fission framework heeft ook erg veel potentieel maar voelt iets minder ''production ready'' aan als OpenFaaS. Binnen de open source community leeft OpenFaaS ook meer dan Fission, desalniettemin heeft het OpenFaaS project ook bijna tienduizend GitHub stars meer wat toch wel wijst op populariteit.
\\\\
Serverless computing is in opmars en dit was ook te merken tijdens het uitwerken van dit onderzoek. Er verschijnen steeds meer bronnen rond het onderwerp op het internet en steeds meer bedrijven gaan ermee aan de slag. De term serverless zorgt nog steeds voor heel wat opschudding bij mensen die geen inzicht hebben in dit onderwerp. Via deze bachelorproef werden de belangrijkste concepten rond serverless nader verklaard in combinatie met enkele demo's die inzicht geven in de mogelijkheden van twee interessante open source serverless frameworks. Na het lezen van dit verslag heeft de lezer hopelijk heel wat nieuwe kennis opgedaan rond serverless en is deze nu ook instaat zich verder te verdiepen in het onderwerp. Er werden twee veelbelovende open source Kubernetes native serverless platformen behandeld, maar er bestaan er nog veel meer die verder onderzoek meer dan waard zijn. Hopelijk kan deze bachelorproef een aanzet geven voor verder onderzoek naar andere, misschien minder bekende maar veelbelovende, open source serverless frameworks.