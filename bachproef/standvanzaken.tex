\chapter{Stand van zaken}
\label{ch:stand-van-zaken}

% Tip: Begin elk hoofdstuk met een paragraaf inleiding die beschrijft hoe
% dit hoofdstuk past binnen het geheel van de bachelorproef. Geef in het
% bijzonder aan wat de link is met het vorige en volgende hoofdstuk.

% Pas na deze inleidende paragraaf komt de eerste sectiehoofding.

Dit hoofdstuk bevat je literatuurstudie. De inhoud gaat verder op de inleiding, maar zal het onderwerp van de bachelorproef *diepgaand* uitspitten. De bedoeling is dat de lezer na lezing van dit hoofdstuk helemaal op de hoogte is van de huidige stand van zaken (state-of-the-art) in het onderzoeksdomein. Iemand die niet vertrouwd is met het onderwerp, weet er nu voldoende om de rest van het verhaal te kunnen volgen, zonder dat die er nog andere informatie moet over opzoeken \autocite{Pollefliet2011}.

Je verwijst bij elke bewering die je doet, vakterm die je introduceert, enz. naar je bronnen. In \LaTeX{} kan dat met het commando \texttt{$\backslash${textcite\{\}}} of \texttt{$\backslash${autocite\{\}}}. Als argument van het commando geef je de ``sleutel'' van een ``record'' in een bibliografische databank in het Bib\TeX{}-formaat (een tekstbestand). Als je expliciet naar de auteur verwijst in de zin, gebruik je \texttt{$\backslash${}textcite\{\}}.
Soms wil je de auteur niet expliciet vernoemen, dan gebruik je \texttt{$\backslash${}autocite\{\}}. In de volgende paragraaf een voorbeeld van elk.

\textcite{Knuth1998} schreef een van de standaardwerken over sorteer- en zoekalgoritmen. Experten zijn het erover eens dat cloud computing een interessante opportuniteit vormen, zowel voor gebruikers als voor dienstverleners op vlak van informatietechnologie~\autocite{Creeger2009}.

\section{Wat is Cloud Computing?}
 
 Cloud computing is alomtegenwoordig en een basiskennis van het begrip is interessant voor iedereen die werkt binnen de IT wereld. In dit hoofdstuk worden allerhande begrippen omtrent cloud computing geïntroduceerd. Na het lezen van dit hoofdstuk zal u in staat zijn om mee te praten over de basiscomponenten en workflows binnen cloud computing. De begrippen in dit hoofdstuk zijn van belang voor het vervolg van dit onderzoek naar serverless of FaaS. 

\subsection{Definitie}

Cloud computing is:
\newline
On-demand computing services die worden aangeboden via het internet. Deze services omvatten onder andere servers, databanken, netwerkfuncties, opslag en nog veel meer. Werken in de cloud volgt het principe: je betaalt voor wat je gebruikt. De cloud biedt flexibiliteit, schaalbaarheid en snelle provisioneer tijd. Het stelt bedrijven in staat benodigde IT infrastructuur uit te besteden en zo dus ook geld te besparen. In de cloud is het mogelijk om up- en down te schalen naargelang de huidige noden, de cloud is met andere woorden elastisch. \autocite{Davis2017}
\newline
\newline
Deze definitie vormt een algemene omschrijving van wat cloud computing allemaal kan inhouden. Het is enigszins zinvol om deze definitie te gebruiken als uitgangspunt om cloud computing verder in detail te omschrijven.

\subsection{Inleiding}
Afgelopen jaren is het internet en de wereld binnen IT enormsnel veranderd en geëvolueerd. Klassieke IT infrastructuur die werken volgens het client-server principe maken de transitie naar een cloudgebaseerde benadering. In een klassieke infrastructuur benadering koopt een bedrijf zelf servers aan, installeert en onderhoudt deze. Een bedrijf moet in dit opzicht alles voorzien: plaats, netwerk, elektriciteit, beveiliging, ... . Een eigen infrastructuur onderhouden in een opgezet datacenter brengt een grote kost met zich mee. In de beginjaren dat bedrijven hun datacenters opstelden werden er vaak voor alle applicaties aparte servers voorzien, met andere woorden er draaide toen één applicatie op één fysieke machine. De traditionele architectuur bracht grote kosten met zich mee en resulteerde vaak ook dat er op sommige momenten te veel resources waren in vergelijking met hoeveel men er maar nodig had. De klassieke benadering had dus heel wat beperkingen. 
\newline
\newline
Later introduceerde VMware als eerste een nieuw product: VMware Virtual Platform, virtualisatie was geboren. Virtualisatie zorgt ervoor dat bovenop de hardware van de server een ''Hypervisor'' kan worden geïnstalleerd. Een hypervisor is een programma dat toelaat om een server onder te verdelen in meerdere servers met elk hun eigen operating system (OS). In de traditionele benadering werd er voor elke legacy-applicatie één server met één OS voorzien. Met gebruik van virtualisatie is het dus mogelijk om meerdere aparte servers te migreren naar één fysieke machine waarop een hypervisor draait. Het migreren van meerdere servers naar één fysieke machine zorgt ervoor dat de resources beter benut worden. Dankzij de hypersvisor draaien meerdere servers nog steeds onafhankelijk van elkaar op hun eigen operating system op dezelfde fysieke hardware. Virtualisatie zorgt ervoor dat er minder fysieke servers  moeten worden aangekocht wat de kost dus aanzienlijk vermindert. Binnen virtualisatie kunnen ook complexe netwerken worden gebouwd zodat gevirtualiseerde servers met elkaar en de buitenwereld kunnen communiceren. \autocite{RedHat2019}. In 
figuur ~\ref{fig:klassiek-vs-virtualisatie} wordt het verschil tussen een klassieke server met één OS en één applicatie vergeleken met een server waarop gevirtualiseerde servers draaien.
\newline
\newline
\begin{figure}
    \caption{Klassieke benadering vs virtualisatie} 
    \includegraphics[width=1\textwidth]{img/klassiek_virtualisatie}
    \label{fig:klassiek-vs-virtualisatie}  
\end{figure}
\newline

Virtualisatie ligt aan de fundamenten van cloud computing, deze technologie maakt het vandaag de dag mogelijk om infrastructuur te draaien in de cloud. Wanneer er naar de cloud gerefereerd wordt, dan bedoelt men meestal de cloud services die grote providers als een Google, Amazon en Microsoft aanbieden. Laten we de conventie maken dat wanneer er in de inleiding gerefereerd wordt naar de cloud, dat dit wijst op de cloud services die door cloud providers worden aangeboden. Later maken we de distinctie tussen welke verschillende soorten cloud er zijn met elk hun eigenschappen. De afgelopen jaren hebben steeds meer bedrijven gekozen om hun applicaties en infrastructuur naar de cloud te migreren. Migratie naar de cloud is interessant in verschillende opzichten, er zijn heel wat voordelen aan verbonden die in sectie ~\ref{voor-en-nadelen} worden uitgelegd. De grootste motivatie waarom bedrijven naar de cloud migreren is enerzijds het feit dat er enkel betaald wordt voor wat men verbruikt en anderzijds neemt het heel wat overhead zoals het onderhouden van infrastructuur weg. Cloud computing zorgt ervoor dat gebruikers een gedeelde poel van resources en opslag kunnen gebruiken bij een cloud provider, deze poel kan worden gezien als duizenden servers waarop virtuele machines of applicaties die worden aangeboden kunnen worden gedraaid. Deze services worden verschaft via het internet. Gebruikers betalen enkel voor de resources die ze verbruiken, zo worden de kosten meestal berekend op de tijd die een server draait en welke gespecificeerde resources deze verbruikt, zo een model wordt ook wel het ''Pay as you go model'' genoemd . De cloud is veelzijdig en biedt mogelijkheden op verschillende niveaus, deze worden in volgende onderdelen besproken.\autocite{Seghal2018} 


\subsection{Cloud types op deployment niveau}
\label{cloud-deployment-level}
De Cloud kan worden onderverdeeld in verschillende types op niveau van deployment. De onderverdeling op deployment level slaat op de locatie waar de cloud infrastructuur draait, bijvoorbeeld in een datacenter van een grote cloud provider of bij een bedrijf op locatie in een eigen datacenter. Volgens \textcite{Goyal2014} kan de cloud worden opgedeeld in vier verschillende soorten met elk hun specifieke eigenschappen. Om te beginnen wordt ook besproken wat men bedoelt met on-premises infrastructuur.

\subsubsection{On-premises}
Een on-premises infrastructuur wijst op een IT infrastructuur die wordt beheerd door een organisatie zelf, de apparaten bevinden zich ook op locatie bij het bedrijf. Een organisatie die over een on-premises of op locatie infrastructuur beschikt staat in voor het volledige beheer hiervan, dit gaat van fysieke installatie van apparaten netwerk enzovoort tot het beveiligen van applicatie op niveau van software. Een on-premises installatie zien we vaak terug in klassieke benaderingen of in bedrijven waar ze nog niet overtuigd van het hele cloud gebeuren. Sommige organisaties kiezen expliciet om hun infrastructuur niet in de cloud te draaien om allerhande redenen gerelateerd aan privacy en confidentialiteit van data.

\subsubsection{Publieke cloud}
De publieke Cloud wordt onderhouden door een derde partij. Publieke cloud providers bieden services en resources aan die op basis van een soort huurcontract aan externen worden verschaft. Klanten die services of infrastructuur huren bij publieke cloud providers kunnen deze raadplegen via het internet. Publieke cloud services worden verschaft aan iedereen, ze zijn met andere woorden voor iedereen toegankelijk. De publieke cloud stelt organisaties in staat te besparen op aankoop van infrastructuur. Het is mogelijk om te betalen naargelang de resources of diensten die worden gebruikt en dit is een heel interessant aspect van de publieke cloud. Data die gemaakt of opgeslagen wordt door gebruikers bevindt zich ook in het datacenter van de cloud provider. Voorbeelden van publieke cloud providers zijn, zoals eerder al aangehaald, Google, Amazon en Microsoft.

\subsubsection{Private cloud}
Private cloud bestaat in verschillende opzichten, het kan een datacenter zijn dat op locatie staat en onderhouden wordt of een cloud infrastructuur zijn die is opgezet in een datacenter waar servers worden gehuurd. Een private cloud wordt door de organisatie zelf volledig onderhouden, er is ook enkel toegang voor de organisatie zelf of voor toegestane derde partijen. Wanneer er gekozen wordt voor een private cloud infrastructuur dan brengt dit de veelzijdigheid en tools van cloud computing met zich mee, het is mogelijk dezelfde tools uit de publieke cloud op te zetten in een private cloudomgeving. Private cloud wordt vooral gekozen voor het waarborgen van privacy en veiligheid van data. Deze benadering wordt door veel bedrijven gekozen die veel confidentiële- en bedrijfskritische data moeten verwerken, denk maar aan banken, overheid en farmaceutica.

\subsubsection{Hybride cloud}
Een hybride cloud bestaat uit minstens één private en één publieke cloud. Wanneer verschillende soorten cloud met elkaar worden samengebracht wordt er verwezen naar een hybride cloud. Een hybride cloud wordt opgezet volgens verschillende standaarden en cloud- en apparatuur specifieke patenten. Een hybride cloud biedt de veelzijdigheid voor up- en down schaling zoals in de publieke cloud, en de veiligheid en integriteit zoals in de private cloud. Het implementeren van een hybride cloud omgeving is moeilijker omdat security hier heel wat overhead met zich meebrengt.

\subsubsection{Communtiy cloud}
Community cloud valt tussen de publieke en private cloud. Een community cloud bestaat uit twee of meer deelnemende partijen. Dit type van cloud valt te vergelijken met de private cloud die gedeeld wordt door meerdere partijen. Gebruik van een community cloud reduceert de aankoopkost van infrastructuur en de management kosten voor het opzetten van het cloud datacenter.


\subsection{Cloud types op service niveau}
\label{cloud-service-level}
Naast onderscheid op basis van de locatie waar cloud infrastructuur gedeployed wordt, maakt men ook onderscheid op niveau van service. Termen zoals PaaS, IaaS en SaaS horen tot deze sectie \autocite{Goyal2014}, in volgende drie secties worden deze veelvoorkomende klassieke cloud computing modellen samengevat.

\subsubsection{Infrastructure-as-a-Service (IaaS)}
Het IaaS service model biedt gebruikers computing, processing, networking en opslag aan waarop applicaties en dergelijke kunnen worden gedraaid. De gebruiker heeft de mogelijkheid om bovenop hardware die voorzien wordt door de cloud provider in te staan voor het volledige beheer van de infrastructuur. De componenten die aanpasbaar zijn, zijn onder andere het OS, de middleware, runtime, data en applicaties. De cloud provider staat in voor de networking, opslag, fysieke servers en bovenliggende virtualisatie. IaaS vormt de basis waarop Cloud computing is gebouwd, overige service modellen bouwen hierop ook verder. Gebruikers die controle willen hebben over (bijna) alle onderdelen van hun cloud infrastructuur opteren voor het gebruik van IaaS. Voorbeelden van IaaS zijn onder andere Google Cloud Platform Compute Engine, Amazon Web Services EC2 en Microsoft Azure.

\subsubsection{Platform-as-a-Service (PaaS)}
PaaS is vooral gericht op ontwikkelaars. Platform-as-a-Service stelt gebruikers in staat op een eenvoudige manier softwareapplicaties op te zetten zonder zorgen over de onderliggende infrastructuur. Cloud providers die PaaS diensten verschaffen staan in voor het volledige beheer van de infrastructuur zoals servers, zowel virtueel als fysiek, computing, opslag, netwerk, beveiliging, bijhorende tools en API's. Het beheer van servers zoals OS en dergelijke wordt ook door de cloud provider voorzien, PaaS voorziet een platform om applicaties op te draaien. Gebruikers staan enkel in voor ontwikkeling van de applicatie en de data die daar bijhoort. Als PaaS oplossing wordt vaak AWS Elastic Beanstalk gebruikt.

\subsubsection{Software-as-a-Service (SaaS)}
In het Software-as-a-Service model wordt de gehele infrastructuur inclusief applicatie beheerd door de provider. SaaS voorziet applicaties die draaien in de cloud en raadpleegbaar zijn via een computer of mobiel apparaat, vaak simpelweg via een webbrowser. In het verleden moest software meestal worden aangekocht en worden geïnstalleerd op individuelen systemen, SaaS is hiervoor de oplossing. Wanneer er gebruik wordt gemaakt van SaaS wordt de software aangerekend ob basis van subscripties per gebruiker, dit kan op basis van tijd of verbruik. SaaS draait, zoals eerder gezegd, in de cloud en vereist dus geen extra installatie van software. Eén van de bekendste SaaS applicaties is Office 365, deze Microsoft services werken met een maandelijkse subscriptie die betaald wordt per gebruiker. Een voorbeeld van een gratis SaaS applicatie is bijvoorbeeld Google Docs.
\begin{figure}
    \caption{Verschillen in cloud types op niveau van service} 
    \includegraphics[width=1\textwidth]{img/cloud_service_level.png}
    \label{fig:cloud-service-levels}  
\end{figure}
\newline

\subsection{Voor-en nadelen}
\label{voor-en-nadelen}
Cloud computing kent voordelen die heel wat mogelijkheden bieden ten opzichte van de klassieke benadering omtrent infrastructuur. \autocite{Azure2019} De voordelen zijn veelbelovend maar ook de nadelen mogen zeker niet over het hoofd worden gezien en dienen in kaart te worden gebracht om op de hoogte te zijn van mogelijke gevaren en bedreigingen.\autocite{Sosinsky2011} 

\subsubsection{Voordelen}

\begin{description}[style=unboxed, labelwidth=\linewidth, listparindent =0pt]
    \item[Lagere kosten]
    Cloud computing zorgt ervoor dat bedrijven zelf geen fysieke hardware meer hoeven aan te kopen voor hun eigen datacenter om software en applicaties te draaien. Een bedrijf hoeft bijvoorbeeld zelf geen webserver meer aan te kopen en te installeren om een webapplicatie te kunnen draaien, de benodigde hardware is beschikbaar bij verschillende cloud providers. Daarnaast zijn er ook geen kosten voor elektriciteitsvoorzieningen. De enige kost zijn de prijzen die de cloud provider aanrekent om de apparatuur of diensten ter beschikking te stellen.
    \newline

    \item[Performantie]
    Datacenters van grote cloud providers bevinden zich verspreid doorheen de hele wereld. De verspreiding van datacenters zorgt voor een lage latency en zorgt ervoor dat werken in de cloud voelt alsof de server in het eigen bedrijf staat. Computing componenten worden ook geregeld geüpgrade naar de laatste versies zodat de performantie verzekerd kan worden.
    \newline

    \item [Veiligheid]
    Doordat servers en opslag niet meer op locatie staan bij bedrijven wordt de kans op datadiefstal al sterk gereduceerd. Er is geen fysieke toegang meer tot infrastructuur en dus zal het onmogelijk zijn om via deze weg data te stelen. Daarnaast zijn datacenters uitgerust met dedicated beveiligingsapparatuur die in vele bedrijven niet aanwezig is. Datacenters zelf zijn erg goed beveiligd, het is haast onmogelijk deze gebouwen te betreden. Op softwareniveau bieden cloud providers ook heel wat diensten aan die helpen bij het beveiligen van applicaties.
    \newline

    \item [Schaalbaarheid en flexibiliteit]
    Cloud computing stelt in staat om up- en down te schalen volgens de noden. Dit wilt zeggen dat een bedrijf eender wanneer kan opteren om meer of minder middelen te huren naar gelang wat nodig is. Cloud computing is met andere woorden elastisch. Een bedrijf zoals Tomorrowland kan bijvoorbeeld doorheen het jaar, wanneer er geen ticketverkoop aan de gang is, de requests naar hun webservers bolwerken met 10 operationele servers. Indien er een ticketverkoop start kan het bedrijf kiezen om op te schalen naar 300 webservers zodat alle requests kunnen worden behandelt zonder dat dit veel moeite kost. De nood voor 30 keer meer servers is, wanneer deze manueel worden geïnstalleerd en opgezet op locatie, een taak die heel veel tijd en werk in beslag neemt maar die in de cloud zo is verwerkt. Dit voorbeeld toont aan dat de cloud mogelijkheden biedt om op een eenvoudige manier te schalen zonder dat dit handenvol geld kost (infrastructuur, tijd).
    \newline
  
\end{description}

\subsubsection{Nadelen}


\begin{description}[style=unboxed, labelwidth=\linewidth, listparindent =0pt]
        \item[Latency]
        Wanneer datacenters ver liggen vanwaar de servers worden geraadpleegd. Bijvoorbeeld je gebruikt thuis, in België, een applicatie die draait in een datacenter in de Verenigde staten, dan kan er latency optreden. Latency is vertraging in datacommunicatie tussen twee systemen, ook wel bekend als ''lag''. Dit probleem begint stilaan te verdwijnen aangezien alle grote cloud providers datacenters verdeeld hebben over de hele wereld.
        \newline
        
        \item[Privacy en security]
        Data legt meestal een langere weg af tussen een klant en cloud provider wanneer er gebruik wordt gemaakt van cloud services dan wanneer die klant zelf gebruik maakt van een infrastructuur op locatie. De langere route die de data neemt brengt ook meer gevaar met zich mee, hoe verder data moet reizen, hoe meer kans op onderschepping. Daarnaast is er vaak geen garantie of de overheid niet meekijkt naar de data die zich in datacenters bevindt bij publieke cloud providers. Datacenters zijn voor hackers ook grote doelwitten omdat ze met mogelijke inbraken hier veel mensen mee kunnen treffen en gegevens kunnen stelen.
        \newline
        
        \item [Vendor lock-in]
        Wanneer er gekozen wordt om gebruik te maken van diensten die een bepaalde cloud provider aanbiedt dan zit je vast aan die cloud provider en is overstappen met je hele infrastructuur vaak een zware klus. Kiezen om over te stappen naar een andere provider brengt mogelijks heel wat complexiteit en problemen met zich mee.
        \newline
        
        
        \item [Afhankelijk van netwerkconnectiviteit]
        Bij gebruik van de publieke cloud is er steeds afhankelijkheid van netwerkconnectiviteit. Wanneer een internetverbinding niet mogelijk is door eventuele problemen bij de Internet Service Provider (ISP) dan kan de cloud niet bereikt worden en zorgt dit voor grote problemen wanneer een bedrijf afhankelijk is van alles wat in de cloud draait. Daarnaast is het ook van belang dat er een snelle netwerkverbinding met voldoende bandbreedte beschikbaar is om effectief en efficiënt gebruik te maken van de cloud.
        \newline
        
        \item [Downtime]
        Downtime is een nadeel dat nog voor problemen kan zorgen. Wanneer een infrastructuur gedraaid wordt binnen een bepaald datacenter en dit gaat neer door interne problemen (technische problemen, onderhoud, gefaalde update/upgrade, brand, natuurramp, ...) dan zijn de infrastructuur en diensten (tijdelijk) niet meer bereikbaar. In het allerslechtste geval kunnen zo ook alle gegevens verloren raken wanneer er geen failover voorzien werd naar een ander datacenter (bijvoorbeeld in geval van een natuurramp). De kans dat dit voorkomt is echter nihil.
        
        
\end{description}