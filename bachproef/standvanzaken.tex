\chapter{Stand van zaken}
\label{ch:stand-van-zaken}

% Tip: Begin elk hoofdstuk met een paragraaf inleiding die beschrijft hoe
% dit hoofdstuk past binnen het geheel van de bachelorproef. Geef in het
% bijzonder aan wat de link is met het vorige en volgende hoofdstuk.

% Pas na deze inleidende paragraaf komt de eerste sectiehoofding.

Dit hoofdstuk bevat je literatuurstudie. De inhoud gaat verder op de inleiding, maar zal het onderwerp van de bachelorproef *diepgaand* uitspitten. De bedoeling is dat de lezer na lezing van dit hoofdstuk helemaal op de hoogte is van de huidige stand van zaken (state-of-the-art) in het onderzoeksdomein. Iemand die niet vertrouwd is met het onderwerp, weet er nu voldoende om de rest van het verhaal te kunnen volgen, zonder dat die er nog andere informatie moet over opzoeken \autocite{Pollefliet2011}.

Je verwijst bij elke bewering die je doet, vakterm die je introduceert, enz. naar je bronnen. In \LaTeX{} kan dat met het commando \texttt{$\backslash${textcite\{\}}} of \texttt{$\backslash${autocite\{\}}}. Als argument van het commando geef je de ``sleutel'' van een ``record'' in een bibliografische databank in het Bib\TeX{}-formaat (een tekstbestand). Als je expliciet naar de auteur verwijst in de zin, gebruik je \texttt{$\backslash${}textcite\{\}}.
Soms wil je de auteur niet expliciet vernoemen, dan gebruik je \texttt{$\backslash${}autocite\{\}}. In de volgende paragraaf een voorbeeld van elk.

\textcite{Knuth1998} schreef een van de standaardwerken over sorteer- en zoekalgoritmen. Experten zijn het erover eens dat cloud computing een interessante opportuniteit vormen, zowel voor gebruikers als voor dienstverleners op vlak van informatietechnologie~\autocite{Creeger2009}.

\section{Wat is Cloud Computing?}
 
 Vooraleer het onderwerp ''serverless'' kan behandelt worden is het van belang om een aantal basisconcepten en begrippen in verband met Cloud computing uit de doeken te doen. Volgende paragrafen geven inzicht in enkele basisprincipes en trachten een begrijpbaar beeld rond Cloud computing te vormen.

\subsection{Definitie}

Cloud computing kan worden omschreven als on-demand computing services die worden aangeboden of verschaft via het internet. Deze services omvatten onder andere servers, databanken, netwerkfuncties, opslag en nog veel meer. Werken in de cloud volgt het principe: je betaalt voor wat je gebruikt. De cloud biedt flexibiliteit, schaalbaarheid en snelle provisioneer tijd. Het stelt bedrijven in staat benodigde IT infrastructuur uit te besteden en zo dus ook geld te besparen. In de cloud is het mogelijk om up- en down te schalen naargelang de huidige noden, de cloud is met andere woorden elastisch. \autocite{Davis2017}

\subsection{Voor-en nadelen}
Cloud computing kent voordelen \autocite{Azure2019} die heel wat mogelijkheden bieden ten opzichte van de klassieke benadering omtrent infrastructuur. De voordelen zijn veelbelovend maar ook de nadelen \autocite{Larkin2018} mogen zeker niet over het hoofd worden gezien en dienen in kaart te worden gebracht om op de hoogte te zijn van mogelijke gevaren en bedreigingen.

\subsubsection{Voordelen}

Lagere kosten
\begin{itemize}
    \item Cloud computing zorgt ervoor dat bedrijven zelf geen fysieke hardware meer hoeven aan te kopen voor hun eigen datacenter om software en applicaties op te draaien. Een bedrijf hoeft bijvoorbeeld zelf geen webserver meer aan te kopen en te installeren om een webapplicatie te kunnen draaien, de benodigde hardware is beschikbaar bij verschillende cloud providers. Daarnaast zijn er ook geen kosten voor elektriciteitsvoorzieningen. De enige kost zijn de prijzen die de cloud provider aanrekent om de apparatuur ter beschikking te stellen.
\end{itemize}

Performantie
\begin{itemize}
    \item Datacenters van grote cloud providers bevinden zich verspreid doorheen de hele wereld. De verspreiding van datacenters zorgt voor een lage latency en zorgt ervoor dat werken in de cloud voelt alsof de server in het eigen bedrijf staat. Computing componenten worden ook geregeld geüpgrade naar de laatste versies zodat de performantie verzekerd kan worden.
\end{itemize}

Veiligheid
\begin{itemize}
    \item Doordat servers en opslag niet meer op locatie staan bij bedrijven wordt de kans op datadiefstal al sterk gereduceerd. Er is geen fysieke toegang meer tot infrastructuur en dus zal het onmogelijk zijn om via deze weg data te stelen. Daarnaast zijn datacenters uitgerust met dedicated beveiligingsapparatuur die in vele bedrijven niet aanwezig is. Datacenters zelf zijn erg goed beveiligd, het is haast onmogelijk deze gebouwen te betreden. Op softwareniveau bieden cloud providers ook heel wat diensten aan die helpen bij het beveiligen van applicaties.
\end{itemize}

Schaalbaarheid en flexibiliteit
\begin{itemize}
    \item Cloud computing stelt in staat om up- en down te schalen volgens de noden. Dit wilt zeggen dat een bedrijf eender wanneer kan opteren om meer of minder middelen te huren naar gelang wat nodig is. Cloud computing is met andere woorden elastisch. Een bedrijf zoals Tomorrowland kan bijvoorbeeld doorheen het jaar, wanneer er geen ticketverkoop aan de gang is, de requests naar hun webservers bolwerken met 10 operationele servers. Indien er een ticketverkoop start kan het bedrijf kiezen om op te schalen naar 100 webservers zodat alle requests kunnen worden behandelt zonder dat dit veel moeite kost. De nood voor 10 keer meer servers is, wanneer deze manueel worden geïnstalleerd en opgezet op locatie, een taak die heel veel tijd en werk in beslag neemt maar die in de cloud zo is verwerkt. Dit voorbeeld toont aan dat de cloud mogelijkheden biedt om op een eenvoudige manier te schalen zonder dat dit handenvol geld kost (infrastructuur, tijd).
\end{itemize}

\subsubsection{Nadelen}