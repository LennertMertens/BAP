%%=============================================================================
%% Voorwoord
%%=============================================================================

\chapter*{Woord vooraf}
\label{ch:voorwoord}

Voor het behalen van de graad Bachelor in de Toegepaste Informatica wordt er verwacht dat de student in staat is een concreet onderzoek uit te werken rond een actueel onderwerp. Na twee en half jaar studeren aan de HoGent was het dit jaar mijn beurt om een bachelorproef te schrijven. Ik heb gekozen mijn bachelorproef te schrijven over serverless computing, specifiek een onderzoek naar een interessant open source serverless framework. Het onderwerp werd gekozen uit persoonlijke interesse in cloud computing en de nieuwe oplaaiende trend, serverless computing. In mijn stage kom ik dagelijks in contact met cloud computing en daarom vond ik het zinvol een bachelorproef rond dit onderwerp te schrijven. Dit onderzoek werd gevoerd voor het bedrijf Nubera, hier heb ik eveneens mijn stage gelopen. De bachlorproef heeft mij persoonlijk heel veel bijgeleerd omdat het hele cloud computing gebeuren relatief abstract en nieuw voor me was. 
\\\\
Als eerste wens ik mijn promotor, Steven Van Impe, te bedanken voor de geleverde feedback en bijsturing wanneer ik dit nodig had. Meneer Van Impe stond altijd klaar om mijn vragen te beantwoorden en dat heeft me veel vooruitgeholpen in het proces.
\\\\
Vervolgens wil ik ook een woord van dank richten tot mijn co-promotor, Pieter-Jan Saveyn. Pieter-Jan heeft me inhoudelijk en op technisch vlak steeds bijgestaan en verbeteringen aanbevolen.
\\\\
Daarnaast wil ik ook mijn vriendin en mijn ouders bedanken die me steeds bleven motiveren wanneer ik het even niet meer zag zitten. Ik wil hun ook bedanken voor de aanbevelingen en opmerkingen die ze hebben aangereikt.
\\\\
Als laatste wil ik ook mijn collega's en vrienden bedanken die steeds grote interesse toonden en steeds bereid waren om na te lezen of te helpen indien dit nodig was.
\\\\
Alle mensen die me hebben bijgestaan verdienen het dubbel en dik hierin vermeld te staan, zonder hen zou deze bachelorproef nooit de hoedanigheid hebben kunnen aannemen zoals deze vandaag is.
