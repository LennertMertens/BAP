%%=============================================================================
%% Methodologie
%%=============================================================================

\chapter{Methodologie}
\label{ch:methodologie}

%% TODO: Hoe ben je te werk gegaan? Verdeel je onderzoek in grote fasen, en
%% licht in elke fase toe welke stappen je gevolgd hebt. Verantwoord waarom je
%% op deze manier te werk gegaan bent. Je moet kunnen aantonen dat je de best
%% mogelijke manier toegepast hebt om een antwoord te vinden op de
%% onderzoeksvraag.

\section{Voorbereiding}
Als eerste heb ik mij de eerste weken van het tweede semester verdiept in het onderwerp, vooral door YouTube video's te kijken maar ook door artikels en blogs te lezen. Ik heb mij ook  verdiept in cloud computing aan de hand van een course op Pluralsight waarin de basisconcepten van cloud computing werden uitgelegd. Ik had nog niet echt kennis van cloud computing en totaal geen kennis van serverless dus geregeld hier wat rond lezen of bekijken hielp mij wel om een beter beeld te krijgen van het volledige plaatje. 

\section{Stand van zaken}
Vervolgens ben ik gestart met het schrijven van mijn literatuurstudie. In de literatuurstudie start ik met een uiteenzetting van cloud computing concepten en principes om vervolgens verder te gaan op serverless. Het volledige serverless plaatje wordt behandeld en gecapteerd aan de hand van duidelijke voorbeelden met afbeeldingen die verschillende componenten visueel voorstellen. De literatuurstudie is geschreven op basis van boeken omtrent cloud computing en serverless infrastructuur. Vanuit wetenschappelijke artikels werden juiste feiten gehaald om vervolgens zo nauwkeurig mogelijke artikels, conferenties en documentatie te verzamelen die inzicht geven in het onderwerp.

\section{Uitwerking}
Bij het schrijven van hoofdstuk \ref{ch:uitwerking} wordt de zoektocht naar 
 serverless frameworks, die in aanmerking komen voor deze bachelorproef, beschreven. Het hoofdstuk vat aan met een inleiding waarin wordt uitgelegd waarnaar er op zoek wordt gegaan. De zoektocht naar mogelijke open source frameworks is vertrokken vanuit een requirements analyse waarin alle requirements opgelegd door Nubera in vermeld staan. Op basis van enkele requirements werden frameworks die in aanmerking komen gekozen. Vervolgens werden de frameworks aan de hand van de requirements ten opzichte van elkaar vergeleken en werd er een lijst opgemaakt waarin de meest veelbelovende frameworks werden gerangschikt van boven naar onder. Op basis van de rangschikking werd het meest veelbelovende framework gekozen samen met Fission, dit was reeds vastgelegd door Nubera. Het gekozen framework en Fission vormen samen de short list. In de short list werd verder onderzoek gedaan naar de frameworks, wat ze inhouden en hoe de architectuur ervan eruitziet. Na de zoektocht worden beide frameworks opgezet in een Proof of Concept om deze tegenover elkaar af te wegen.

\section{Proof of Concept}
