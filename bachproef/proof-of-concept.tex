\chapter{Proof of Concept}
\label{ch:proof-of-concept}
Dit hoofdstuk bevat het vergelijkend experiment tussen twee eerder gekozen open source serverless (FaaS) frameworks, namelijk Fission en OpenFaaS. In dit onderdeel worden de twee frameworks opgezet op een Kubernetes cluster die bestaat uit één node. Beide frameworks worden opgezet op een MacBook Pro waarop Minikube draait. Minikube is een ''Out-of-the-box'' Kubernetes cluster die overal kan draaien. Er wordt gekozen om gebruik te maken van Minikube omdat op deze manier beide frameworks op identiek dezelfde hardware kunnen gedraaid worden. Daarnaast biedt Minikube dezelfde functionaliteiten als een volledige Kubernetes cluster die elders is opgezet. Dit hoofdstuk is opgedeeld in twee grote onderdelen waarbinnen telkens dezelfde secties met dezelfde stappen voor elk framework terug te vinden zijn.

\section{Onderliggende hardware}
Het experiment zal worden uitgevoerd op een MacBook Pro, model 2018 met volgende specificaties:
\begin{itemize}
    \item Processor: 2,2 GHz Intel Core i7
    \item Memory: 16 GB 2400 MHz DDR4
    \item Graphics: Radeon Pro 555X 4 GB en Intel UHD Graphics 630 1536 MB
    \item Opslag: 256 GB SSD
\end{itemize}
\section{Klaarzetten omgeving}

\section{OpenFaaS}
In deze sectie wordt de opstelling met OpenFaaS beschreven. Het experiment wordt opgezet op een MacBook Pro met specificaties zoals eerder beschreven. De stappen die werden doorlopen worden stapsgewijs verklaard.

\subsection{Opzetten Minikube}




\section{Fission}